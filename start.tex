\documentclass[14pt, a4paper]{article}

\usepackage{extsizes}

\usepackage{indentfirst} %первый абзац будет начинаться с красной строки
\setlength{\parindent}{1.25cm} %установка размера отступа для красной строки

%шрифты
%\usepackage{pscyr}
\renewcommand{\rmdefault}{ftm}% Times New Roman

\usepackage[utf8]{inputenc}
\usepackage[T2A]{fontenc}
\usepackage[russian]{babel}

\pretolerance=10000 % отменяет переносы (выглядит как выравнивание по ширине)

%нумерация формул по разделам
\usepackage{chngcntr}
%\counterwithin{equation}{section}

% загружаем пакеты
\usepackage{floatrow}
\usepackage{cite,enumerate,float,indentfirst}
\usepackage{amsmath, amsfonts, amssymb, amsthm}
\usepackage{amssymb}
\usepackage{amsmath}%написание текста в формулах
\usepackage{amsbsy}

\usepackage[pdftex]{lscape}%для альбомной страницы
\usepackage{graphicx}
\usepackage{epsfig}
\usepackage{float}
\usepackage{xcolor}

\usepackage{setspace}

\linespread{1.06} % 18пт интервал


%настройка разделов
\makeatletter
\renewcommand\section{\@startsection {section}{1}{1.25cm}%
                                   {-3.5ex \@plus -1ex \@minus -.2ex}%
                                   {2.3ex \@plus.2ex}%
                                   {\normalfont\fontsize{15}{15}\bfseries}}

\renewcommand\subsection{\@startsection {subsection}{2}{1.25cm}%
                                   {-3.5ex \@plus -1ex \@minus -.2ex}%
                                   {2sp}% No space after subsections
                                   {\normalfont\fontsize{15}{15}\bfseries}}

\renewcommand\subsubsection{\@startsection {subsubsection}{3}{1.25cm}%
                                   {-3.5ex \@plus -1ex \@minus -.2ex}%
                                   {2sp}% No space after subsections
                                   {\normalfont\fontsize{15}{15}\bfseries}}
\makeatother



\frenchspacing
\sloppy
\usepackage{ragged2e}
\justifying
\graphicspath{{images/}}
\restylefloat{table}

% Поля страницы
\usepackage{geometry}
\geometry{left=3cm}
\geometry{right=1cm}
\geometry{top=2cm}
\geometry{bottom=2cm}

\renewcommand{\partname}{Глава}

\usepackage[pdftex]{hyperref}%ссылки

\makeatletter

\renewcommand{\@biblabel}[1]{#1.} %нумерация в списке литературы без квадратных скобок
\makeatother

\renewcommand\small{\fontsize{13}{10}\selectfont}
\renewcommand\large{\fontsize{15}{10}\selectfont}
\usepackage[textfont=small]{caption}

\DeclareCaptionLabelFormat{gostfigure}{\small{Рисунок} #2}
%%рисунок
\usepackage[tableposition=top]{caption}
\usepackage{caption}
%%подпись с дефисом
\RequirePackage{caption}
\DeclareCaptionLabelSeparator{defffis}{--}
\captionsetup{justification=centering, labelsep=defffis}


%%Оглавление начало
\usepackage{tocloft}
\renewcommand{\cfttoctitlefont}{\hspace{0.38\textwidth} \bfseries\MakeUppercase}
\renewcommand{\cftbeforetoctitleskip}{-1em}

\renewcommand{\cftsecleader}{\bfseries\cftdotfill{\cftdotsep}}
\renewcommand{\cftsecfont}{\normalsize\bfseries}
\renewcommand{\cftsubsecfont}{\hspace{31pt}}
\renewcommand{\cftsubsubsecfont}{\hspace{11pt}}
\renewcommand{\cftbeforesecskip}{1em}
\renewcommand{\cftparskip}{-1mm}
\renewcommand{\cftdotsep}{1}
\renewcommand{\cftaftertoctitleskip}{20pt}
\setcounter{tocdepth}{4} % для оглавления выводить n=3 это chapter, section, subsection, subsubsection;

\renewcommand{\cftsecfont}{\normalsize\normalfont} %\MakeUppercase{\secname} }
\renewcommand\cftsecpagefont{\normalfont}


\cftsetindents{section}{-1.5cm}{1.15cm}
\cftsetindents{subsection}{-2cm}{1.15cm}
\cftsetindents{subsubsection}{-0.5cm}{1.5cm}

\usepackage{etoolbox}
\makeatletter
\pretocmd{\section}{\addtocontents{toc}{\protect\addvspace{20\p@}}}{}{}
\pretocmd{\subsection}{\addtocontents{toc}{\protect\addvspace{10\p@}}}{}{}
\pretocmd{\subsubsection}{\addtocontents{toc}{\protect\addvspace{10\p@}}}{}{}
\makeatother


\newenvironment{enumerate*}%
  {\begin{enumerate}%
    \setlength{\itemsep}{0.5pt}%
    \setlength{\parskip}{0.5pt}}%
  {\end{enumerate}} %списки

%уменьшение расстояние для библиографических источников
\let\oldthebibliography=\thebibliography
\let\endoldthebibliography=\endthebibliography
\renewenvironment{thebibliography}[1]{%
\begin{oldthebibliography}{#1}% \setlength{\parskip}{0ex}%
\setlength{\itemsep}{0ex}%
}%
{%
\end{oldthebibliography}%
}

\usepackage{chngcntr}
\hyphenpenalty=1000%вроде как переносы не разрешает ставить


\renewcommand{\cftdotsep}{0.5}



\begin{document}

\counterwithin{table}{section}
\renewcommand{\figurename}{Рисунок}
\renewcommand\contentsname{\large{ОГЛАВЛЕНИЕ}}
\renewcommand\refname{\centering СПИСОК ИСПОЛЬЗОВАННЫХ ИСТОЧНИКОВ}
\renewcommand\refname{\centering СПИСОК ЛИТЕРАТУРЫ}

\section{Решение}
Составим уравнения естественной конвекции. При этом сделаем следующие допущения: 

\begin{enumerate}[(a)]
\item Жидкость несжимаемая, т.е. плотность жидкости не зависит от давления.
\item В объеме жидкости отсутствуют источники и стоки тепла.
\end{enumerate}

Математическую модель процесса естественной конвекции обычно составляют из уравнения Навье-Стокса (закона сохранения количества движения):

\begin{equation}
\begin{gathered}
\label{f1}
\rho[\frac{\partial \overline \nu}{\partial t} + (\overline \nu \cdot \nabla)\overline \nu]=-\nabla p + \eta \nabla^2 \overline \nu + \frac{1}{3} \eta \nabla (\nabla \cdot \overline \nu) + \rho \overline g
\end{gathered}
\end{equation}

уравнения неразрывности (закона сохранения массы):

\begin{equation}
\begin{gathered}
\label{f2}
\frac{\partial \rho}{\partial t} + \nabla \cdot (\rho \overline \nu) = 0
\end{gathered}
\end{equation}

уравнения переноса тепла (закона сохранения энергии):

\begin{equation}
\begin{gathered}
\label{f3}
c \rho(\frac{\partial T}{\partial t} + \overline \nu \cdot \nabla T) = \lambda \nabla^2 T
\end{gathered}
\end{equation}

и уравнение состояния:

\begin{equation}
\begin{gathered}
\label{f4}
\rho = \rho (T)
\end{gathered}
\end{equation}

где неизвестные:

\begin{enumerate}
\item $ \vec v = (v_x, v_y, v_z)$
\item $ p = p(x,y,z,t)$
\item $ T = T(x,y,z,t)$
\end{enumerate}

и параметры:

\begin{enumerate}
\item динамическая вязкость $\eta$
\item теплопроводность $\lambda$
\item удельная теплоемкость $c$
\item время $t$
\item напряженность гравитационного поля $\overline g$ 
\end{enumerate}

Обратимся к приближению Буссинеска-Обербека:

Все параметры --- константы

$$ \rho = \rho_0 = \rho (T_0) = const $$

$T_0$ выбираем в области задачи. Рассмотрим $\rho(T)$, раскладывая ее в ряд Тейлора в окрестности значения $T_0$

$$\rho (T)) = \rho(T_0) + (T - T_0) \frac{\partial \rho(T_0))}{\partial T} + O(\left\lVert T - T_0 \right\rVert ^ 2)$$

$\rho = \rho_0[1 - \beta(T - T_0)]$ --- приближение Буссинеска

где $\beta = - \frac{1}{\rho_0} \frac{\partial \rho(T_0)}{\partial T}$ --- коэффициент теплового объемного расширения жидкости при $T=T_0$

Главная идея приближения Буссинеска-Обербека заключается в том, что зависимость плотности от температуры учитывается лишь в члене с объемной силой тяжести $\rho \overline g$, а в остальных случаях полагают $\rho = \rho_0$. При таких допущениях система \eqref{f1}-\eqref{f4} примет вид:

\begin{equation}
\begin{gathered}
\label{f5}
\frac{\partial \overline \nu}{\partial t} + (\overline \nu \cdot \nabla)\overline \nu = -\frac{1}{\rho_0}\nabla p + \nu \nabla^2 \overline \nu + [1-\beta(T-T_0)]\overline g
\end{gathered}
\end{equation}

\begin{equation}
\begin{gathered}
\label{f6}
\nabla \cdot \overline \nu = 0
\end{gathered}
\end{equation}

\begin{equation}
\begin{gathered}
\label{f7}
\frac{\partial T}{\partial t} + \overline \nu \cdot \nabla T = a \nabla^2 T
\end{gathered}
\end{equation}

В приближении Буссинеско \eqref{f1}-\eqref{f3} соответствует \eqref{f5}-\eqref{f7} если ${\beta |T-T_0| \ll 1}$.

Обратим внимание на то, что функции тока и завихренности в этих уравнениях – это векторные величины. Запишем уравнения в скалярном виде.

\begin{equation}
\begin{gathered}
\label{f8}
\frac{\partial \nu_x}{\partial t} + \nu_x\frac{\partial \nu_x}{\partial x} + \nu_y\frac{\partial \nu_x}{\partial y} + \nu_z \frac{\partial \nu_x}{\partial z} = -\frac{1}{\rho_0}\frac{\partial p}{\partial x} + \nu (\frac{\partial^2 \nu_x}{\partial x^2} + \frac{\partial^2 \nu_x}{\partial y^2} + \frac{\partial^2 \nu_x}{\partial z^2})\\
\frac{\partial \nu_y}{\partial t} + \nu_x\frac{\partial \nu_y}{\partial x} + \nu_y\frac{\partial \nu_y}{\partial y} + \nu_z \frac{\partial \nu_y}{\partial z} = -\frac{1}{\rho_0}\frac{\partial p}{\partial y} + \nu (\frac{\partial^2 \nu_y}{\partial x^2} + \frac{\partial^2 \nu_y}{\partial y^2} + \frac{\partial^2 \nu_y}{\partial z^2})\\
\frac{\partial \nu_z}{\partial t} + \nu_x\frac{\partial \nu_z}{\partial x} + \nu_y\frac{\partial \nu_z}{\partial y} + \nu_z \frac{\partial \nu_z}{\partial z} = -\frac{1}{\rho_0}\frac{\partial p}{\partial z} + \nu (\frac{\partial^2 \nu_z}{\partial x^2} + \frac{\partial^2 \nu_z}{\partial y^2} + \frac{\partial^2 \nu_z}{\partial z^2}) - [1-\beta(T-T_0)]g
\end{gathered}
\end{equation}

\begin{equation}
\begin{gathered}
\label{f9}
\frac{\partial \nu_x}{\partial x} + \frac{\partial \nu_y}{\partial y} + \frac{\partial \nu_z}{\partial z} =0
\end{gathered}
\end{equation}

\begin{equation}
\begin{gathered}
\label{f10}
\frac{\partial T}{\partial t} + \nu_x \frac{\partial T}{\partial x}+ \nu_y \frac{\partial T}{\partial y} + \nu_z + \frac{\partial T}{\partial z} = a(\frac{\partial^2 T}{\partial x^2} + \frac{\partial^2 T}{\partial y^2} + \frac{\partial^2 T}{\partial z^2})
\end{gathered}
\end{equation}

\eqref{f8}-\eqref{f10} --- система уравнений в приближении Буссиенска в естественных переменных.

\end{document}